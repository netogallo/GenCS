\documentclass{article}
\usepackage{amsfonts}
\usepackage{graphicx}
\usepackage{mathtools}
\title{Assignment 6: Abstract Procedures}
\author{Ernesto Rodriguez}
\begin{document}
\maketitle

\section{Problem 2}

\subsection{Part 1}

\begin{tabular}{|c|c|c|}
  \hline
  {\bf Iteration Nr.} & {\bf Call} & {\bf Retrun Value}\\
  \hline
  1 & g(s(s(0)),0) & s(s(0)) \\
  \hline
  2 & g(0,s(0)) & 0 \\
  \hline
  3 & f(0) & 0 \\
  \hline
\end{tabular}
  
\subsection{Part 2}

The recursion relation is as follows:
\[
\begin{array}{ll}
  g=\{\langle 0,1 \rangle,\langle n,n-1\rangle,\langle n,n \rangle\} \\
  f=\{\langle n,n-2 \rangle \}
\end{array}
\]

\subsection{Part 3}
For f there is no infinite chain in the recursion relation so it terminates. For g the function terminates if both inputs are different. The reason is that (0,0) dosent terminate on g and since the function substracts one from every number on each iteration, the same numbers eventually get to (0,0);
\subsection{Part 4}
The `readable` version of the function g would be:
\[
g(x,y) =
\left\{
	\begin{array}{ll}
		?  & \mbox{if } x=y \\
		4*y+(x-y)/2 +1 & \mbox{if } x>y\\
                4*x+(y-x)/2 -1 & \mbox{otherwise}
	\end{array}
\right.
\]

The function f simply halves it's argument.
\end{document}

