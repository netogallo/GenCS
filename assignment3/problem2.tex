\documentclass{article}
\usepackage{amsfonts}
\usepackage{graphicx}
\usepackage{mathtools}
\title{Assignment 4: Relations, Functions and Introducction to SML}
\author{Ernesto Rodriguez}
\begin{document}
\maketitle

\section{Problem 2}

Lets assume the following:

\begin{enumerate}
  \item{Let's assume that the functions that satisfy the given conditions are all functions such that $f(x) \leq x$.}
  \item{Since the function is injective and the natural numbers are countable. Under the first condition we must also assume that all numbers $n \in \mathbb{N}$ in the codomain are mapped by some f(x) $x \in \mathbb{N}$ | $x \leq n$. In other words all numbers smaller than a certain number are mapped by functions of a number smaller or equal than the number.}
\end{enumerate}

Let's now suppose there is some $f(x)$ | $f(x) \leq x$ $\forall$ $x \leq N$. But for some $n > N$, f(n)>n:

\[
f^2(n) \leq \frac{n+f(n)}{2}
\]

For the above to be true, $f^2(n) < f(n)$ since $n<f(n)$. This means $f^2(n)$ must be in the interval [N,f(n)]. Now let's consider the case $f(n)$:

\[
f^3(n) \leq \frac{f(n)+f^2(n)}{2}
\]

We know that for this equivalence to hold:

\[
f^3(n) \leq f(n) \text{ or } f^3(n) \leq f^2(n)
\]

But since we know that $f^2(n) \leq f(n)$ then is evident that $f^3 \leq f(n)$ must be true for the above inequality to hold. Since the function must be injective, the last statement implies that $f^3(n)$ must be in the interval [N,f(n)] as well. We can do this for the next k steps, namely $f^k(x)$ and conclude that $f^k(n)$ must belong to the interval [N,f(n)]. This means we can always choose k to be bigger than f(n)-N and conclude that the function $f(x)$ can't be injective since all the natural numbers in the interval [N,f(n)] will already be mapped by some $f^i(n)$. 

{\bf Conclusion } The proof above states that the set \{f(n)\} of functions that satisfy the inequality above must be defined as follows:

\[
\{f(n)|f:\mathbb{N} \rightarrow \mathbb{N} \wedge f(n)\leq n \wedge \text{ f(n) is injective}\}
\]

The only function in such set is $f(n)=n$.

\end{document}
