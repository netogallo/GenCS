\documentclass{article}
\usepackage{amsfonts}
\usepackage{graphicx}
\usepackage{mathtools}
\title{Assignment 4: Relations, Functions and Introducction to SML}
\author{Ernesto Rodriguez}
\begin{document}
\maketitle

\section{Problem 3}

First lets get the set C of all elements paired with themselves, namely <a,a>. Theese elements, the pair of an element with itself must be included in the relation in order for it to be reflexive.

\[
C=\{<a,a> | a \in \{a,b,c,d,e\}\}
\]

The power set minus the empty set contains all subsets of pairs with the same element, namely $P(C)$. The size is $2^5-1$. 

Lets consider case by case bias:

\begin{enumerate}

  \item{{\bf Sets with one pair: }Theese sets are already reflexive and symetric relations, we have 5 different. And we can't generate any new relations.}
  \item{{\bf Sets with more than one pair: }They are already reflexive and symetric. But we can generate from each one more reflexive and symetric relations by adding to it more pairs. To be exact, we can know the ammount of relations that can be generated by multiplying the ammount of sets with n elements by n. So this boils to:

    \[
    |R|=1*\binom{5}{1}+2*\binom{5}{2}+3*\binom{5}{3}+4*\binom{5}{4}+5*\binom{5}{5}=80
    \]

    The General case for n elements is:

    \[
    |R|=\sum_{i=0}^n i*\binom{n}{i}
    \]
    }
\end{enumerate}

    The answer to the question boils down to 80 relations.

\end{document}
    
