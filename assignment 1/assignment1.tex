\documentclass{article}
\usepackage{amsfonts} 
\usepackage{pst-3dplot} 
\title{Assignment 1: Elementary Math}
\author{Ernesto Rodriguez}
\begin{document}
\maketitle

\section{Notation}

Zero or first unary natural number: 0
Successor of n: s(n)
Multiple successor operatios: sss(n), equals to the successor of the succesor of the successor of n.
Multiplication Operator: *

\section{Problem 1}

{\bf Base Case:}

$(1+x)^n \ge n * x$

{\bf Step Case:}

$(1+x)^{n+1} \ge (n+1) * x$ 

$(1+x)^n * (1+x) \ge x * (n+1) $ 

$(1+x)^n + x * (1+x)^n \ge x * n + x $ 

$ (1+x)^n \ge n * x $ ,Then we have:

$x * (1+x)^n \ge x $

$ (1+x)^n \ge 1 $

\section{Problem 3}

\begin{enumerate}

  {\item {\bf Base Case:}

    $a + b = b + a, b=0$ 
    $a+0=0+a$ 
    $a=0+a$ 
    $a+0=0+a+0$
    $a=0+a$ 

    {\bf Step Case: }

    $a+b=b+a$
    $s(a+b)=s(b+a)$
    $a+s(b)=b+s(a)$
    $a+b+s(0)=b+a+s(0)$ , a+b=b+a So we have: 
    $s(0)=s(0)$
    }

  {\item {\bf Base Case:} 

    $(a+b) * c = a*c + b*c$ ,c=0
    $(a+b) * 0 = a*0 + b*0$
    $(a+b) * 0 = 0$

    {\bf Step Case:}
    $(a+b) * c = a*c + b*c$ ,c=s(c)
    $(a+b) * s(c) = a*s(c) + b*s(c)$
    $(a+b) + (a+b)*s(c) = a + a*c + b + b*c$
    $a+b+a*c+b+b*c=a+a*c+b+b*c$
    }

  {\item {\bf Base Case:}

    $a*0=0*a$
    $a*s(0)=s(0)*a$
    $a+a*0=s(0)*a$
    $a=s(0)*a$

    {\bf Step Case:}

    $a*s(b)=s(b)*a$
    $a*(b+s(0))=(b+s(0))*a$
    $a*b + a*s(0)=b*a +s(0)*a$
    $a*b=b*a$ and $a*s(0)=s(0)*a$
    
  }

  \end{enumerate}

\section {Problem 4}

Let $P_1,P_2,...,P_n$ arbitrary convex polygons.

{\bf Base Case} 

We assume the following:
\begin{itemize}
  {\item $P_1,P_2,P_3$ have a common edge.}
  {\item $P_2,P_3,P_4$ have a common edge.}

\end{itemize}

Then $P_1,P_2,P_3,P_4$ have a common edge. Since $P_2,P_3$ are in both cases.

{\bf Step Case}

Prove the property holds for $P_{s(n)}$
\begin{itemize}
  {\item $P_n,P_{s(n)},P_{ss(n)}$ have a common edge.}
  {\item $P_{s(n)},P_{ss(n)},P_{sss(n)}$ have a common edge.}
\end{itemize}

$P_n,P_{s(n)},P_{ss(n)}$ have a common edge with all the polygons. $P_{s(n)},P_{ss(n)}$ appear in both sets, so $P_{s(n)},P_{ss(n)},P_{sss(n)}$ should also have a common point with all the polygons.
  

\end{document}
